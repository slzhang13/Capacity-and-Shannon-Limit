\documentclass[12pt,a4paper]{article}

\usepackage[slantfont, boldfont]{xeCJK}
\usepackage{graphicx}
\usepackage{fontspec}
\usepackage{hyperref}
\usepackage{amsmath}
\setmainfont{Times New Roman}
\setCJKmainfont{SimSun}

\usepackage{enumitem}
\usepackage{array}
\usepackage{tikz}
\usepackage{caption}
\usepackage{xcolor}
\usepackage{soul}
\usepackage{geometry} 
\usepackage{fancyhdr} 
\usepackage{listings}
\usepackage{amssymb}
\usepackage{bm}

\usepackage{booktabs}
\usepackage[bottom,marginal]{footmisc}
\usepackage{lastpage}

\geometry{left=25mm,right=20mm,top=25mm,bottom=25mm}
\pagestyle{fancy}

\fancyhf{}

\usepackage{indentfirst}
\setlength{\parindent}{2em}

\fancyhead{}
\cfoot{\thepage / \pageref{LastPage}}
\usepackage{float}

\renewcommand{\headrulewidth}{0.1mm}
\renewcommand{\footrulewidth}{0mm}
\usepackage{siunitx}
\usepackage{amssymb}
\usepackage{setspace}

\usepackage{cases}
\hypersetup{hidelinks}
\usepackage{titlesec}

\title{CM-AMI推导}
\author{}
\date{}

\begin{document}

\captionsetup[figure]{labelformat={default},labelsep=period,name={图}}
\captionsetup[table]{name={表},labelsep=period}
\setlength{\baselineskip}{20pt}
\maketitle
\thispagestyle{fancy}

1. 2-D constellation case

Suppose AWGN with $N_o=1$ for derivation simplicity.

\begin{equation}
    \begin{split}
        C&=I(X;Y)\\
        &=\sum_{x\in X}\int_{-\infty}^{+\infty}\int_{-\infty}^{+\infty}p(x,y)\log\frac{p(y\mid x)}{p(y)}dy_{I}dy_{Q}\\
        &=\sum_{x\in X}p(x)\int_{-\infty}^{+\infty}\int_{-\infty}^{+\infty}p(y\mid x)\log\frac{p(y\mid x)}{\sum_{x^\prime\in X}p(y\mid x^\prime)p(x^\prime)}dy_{I}dy_{Q}\\
        &=\sum_{x\in X}p(x)\int_{-\infty}^{+\infty}\int_{-\infty}^{+\infty}\frac{1}{\pi}e^{-|y-x|^2}\log\frac{e^{-|y-x|^2}}{\sum_{x^\prime\in X}e^{-|y-x^\prime|^2}p(x^\prime)}dy_{I}dy_{Q}\\
        &=-\frac{1}{\pi}\sum_{x\in X}p(x)\int_{-\infty}^{+\infty}\int_{-\infty}^{+\infty}e^{-|y-x|^2}\log\frac{\sum_{x^\prime\in X}e^{-|y-x^\prime|^2}p(x^\prime)}{e^{-|y-x|^2}}dy_{I}dy_{Q}\\
        &=-\frac{1}{\pi}\sum_{x\in X}p(x)\int_{-\infty}^{+\infty}\int_{-\infty}^{+\infty}e^{-t_I^2}e^{-t_Q^2}\log{\sum_{x^\prime\in X}e^{|t|^2-|x-x^\prime+t|^2}p(x^\prime)}dt_{I}dt_{Q}\\
        &=-\frac{1}{\pi}\sum_{x\in X}p(x)\int_{-\infty}^{+\infty}\int_{-\infty}^{+\infty}e^{-t_I^2}e^{-t_Q^2}\log{\sum_{x^\prime\in X}e^{|t_I|^2+|t_Q|^2-|x-x^\prime+t_I+i\cdot t_Q|^2}p(x^\prime)}dt_{I}dt_{Q}
        \end{split}\tag{1-1}
\end{equation}

When the constellation symbols are taken with equal probability, the above formula reduces to

\begin{equation}
    \begin{split}
        C&=I(X;Y)\\
        &=m-\frac{1}{M\pi}\sum_{x\in X}\int_{-\infty}^{+\infty}\int_{-\infty}^{+\infty}e^{-t_I^2}e^{-t_Q^2}\log{\sum_{x^\prime\in X}e^{|t_I|^2+|t_Q|^2-|x-x^\prime+t_I+i\cdot t_Q|^2}}dt_{I}dt_{Q}
    \end{split}\tag{1-2}
\end{equation}

where $m=\log_2M$ represents the number of coded bits per constellation symbol.

Notice the calculation of average symbol energy when the constellation symbols are not taken with equal probability.
$\bar{E_s}=\sum_{s\in \chi}p(s)|s|^2$

------



2. 1-D constellation case

Suppose AWGN with $N_o=1$ for derivation simplicity.

\begin{equation}
    \begin{split}
        C&=I(X;Y)\\
        &=\sum_{x\in X}\int_{-\infty}^{+\infty}p(x,y)\log\frac{p(y\mid x)}{p(y)}dy\\
        &=\sum_{x\in X}p(x)\int_{-\infty}^{+\infty}p(y\mid x)\log\frac{p(y\mid x)}{\sum_{x^\prime\in X}p(y\mid x^\prime)p(x^\prime)}dy\\
        &=\sum_{x\in X}p(x)\int_{-\infty}^{+\infty}\frac{1}{\sqrt{\pi}}e^{-|y-x|^2}\log\frac{e^{-|y-x|^2}}{\sum_{x^\prime\in X}e^{-|y-x^\prime|^2}p(x^\prime)}dy\\
        &=-\frac{1}{\sqrt{\pi}}\sum_{x\in X}p(x)\int_{-\infty}^{+\infty}e^{-|y-x|^2}\log\frac{\sum_{x^\prime\in X}e^{-|y-x^\prime|^2}p(x^\prime)}{e^{-|y-x|^2}}dy\\
        &=-\frac{1}{\sqrt{\pi}}\sum_{x\in X}p(x)\int_{-\infty}^{+\infty}e^{-|t|^2}\log{\sum_{x^\prime\in X}e^{|t|^2-|x-x^\prime+t|^2}p(x^\prime)}dt\\
        &=-\frac{1}{\sqrt{\pi}}\sum_{x\in X}p(x)\int_{-\infty}^{+\infty}e^{-t^2}\log{\sum_{x^\prime\in X}e^{t^2-(x-x^\prime+t)^2}p(x^\prime)}dt\\
    \end{split}\tag{2-1}
\end{equation}

When the constellation symbols are taken with equal probability, the above formula reduces to

\begin{equation}
    \begin{split}
        C&=I(X;Y)\\
        &=m-\frac{1}{M\sqrt{\pi}}\sum_{x\in X}\int_{-\infty}^{+\infty}e^{-t^2}\log{\sum_{x^\prime\in X}e^{t^2-(x-x^\prime+t)^2}}dt\\
    \end{split}\tag{2-2}
\end{equation}

where $m=\log_2M$ represents the number of coded bits per constellation symbol.

Notice the calculation of average symbol energy when the constellation symbols are not taken with equal probability.
$\bar{E_s}=\sum_{s\in \chi}p(s)|s|^2$

------



3. 1-D case is a special 2-D case 

\begin{equation}
    \begin{split}
        C&=I(X;Y)\\
        &=-\frac{1}{\pi}\sum_{x\in X}p(x)\int_{-\infty}^{+\infty}\int_{-\infty}^{+\infty}e^{-t_I^2}e^{-t_Q^2}\log{\sum_{x^\prime\in X}e^{|t_I|^2+|t_Q|^2-|x-x^\prime+t_I+i\cdot t_Q|^2}p(x^\prime)}dt_{I}dt_{Q}\\
        \end{split}\tag{3-1}
\end{equation}

When the constellation is 1-D, i.e. $x$ and $x^\prime$ is real, equation (3-1) can be rearranged.

\begin{equation}
    \begin{split}
        C&=I(X;Y)\\
        &=-\frac{1}{\pi}\sum_{x\in X}p(x)\int_{-\infty}^{+\infty}\int_{-\infty}^{+\infty}e^{-t_I^2}e^{-t_Q^2}\log{\sum_{x^\prime\in X}e^{|t_I|^2+|t_Q|^2-|x-x^\prime+t_I|^2+|i\cdot t_Q|^2}p(x^\prime)}dt_{I}dt_{Q}\\
        &=-\frac{1}{\pi}\sum_{x\in X}p(x)\int_{-\infty}^{+\infty}\int_{-\infty}^{+\infty}e^{-t_I^2}e^{-t_Q^2}\log{\sum_{x^\prime\in X}e^{|t_I|^2-|x-x^\prime+t_I|^2}p(x^\prime)}dt_{I}dt_{Q}\\
        &=-\frac{1}{\sqrt{\pi}}\sum_{x\in X}p(x)\int_{-\infty}^{+\infty}e^{-t_I^2}\log{\sum_{x^\prime\in X}e^{|t_I|^2-|x-x^\prime+t_I|^2}p(x^\prime)}dt_{I}\int_{-\infty}^{+\infty}\frac{1}{\sqrt{\pi}}e^{-t_Q^2}dt_{Q}\\
        &=-\frac{1}{\sqrt{\pi}}\sum_{x\in X}p(x)\int_{-\infty}^{+\infty}e^{-t_I^2}\log{\sum_{x^\prime\in X}e^{|t_I|^2-|x-x^\prime+t_I|^2}p(x^\prime)}dt_{I}
        \end{split}\tag{3-2}
\end{equation}

Equation (3-2) is just an equivalent form of equation (2-1).




4. Relationship between several form of signal to noise ratio

\begin{equation}
    SNR=\frac{\sigma^2_s}{\sigma_n^2}=\begin{cases}
        \frac{E_s}{N_o}\cdot2 &\quad\quad \text{1-D constellation}\\
        \\
        \\
        \frac{E_s}{N_o} &\quad\quad \text{2-D constellation}\\
        \end{cases}\tag{4-1}
\end{equation}


\begin{equation}
    \frac{E_s}{N_o}=\frac{E_b*\rho}{N_o}
\tag{4-2}
\end{equation}

where $\rho$ is the data rate in bits per symbol (or channel use) and

\begin{equation}
    \rho=m\cdot R
\tag{4-3}
\end{equation}

where $m$ is the number of coded bits per constellation symbol and $R$ is the code rate of FEC.


% \bibliographystyle{IEEEtran}
% \bibliography{refs}

% https://github.com/gcsxdu/stuff/blob/main/IEEEtran.bst
% refs.bib

\end{document}